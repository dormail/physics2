\documentclass[a4paper]{article}
\usepackage[utf8]{inputenc}
\usepackage{fancyhdr}
\usepackage[margin=2.5cm]{geometry}

\usepackage{amsmath, amssymb}

\fancyhf{}
% vspaces in den headern fuer Distanzen notwendig
% linke Seite: Namen der Abgabegruppe
\lhead{\textbf{Etem Kalyon\\Matthias Maile}\vspace{1.5cm}}
% rechte Seite: Modul, Gruppe, Semester
\rhead{\textbf{Physik II - Gruppe 2\\Sommersemester 2020}\vspace{1.5cm}}
% Center: nr. des blattes
\chead{\vspace{2.5cm}\huge{\textbf{0. Übungsblatt}}}
% benoetigt damit der eigentliche Text nicht in der Überschrift steckt
\setlength{\headheight}{4cm} 

% newcommands für diese abgabe
\newcommand{\nabvec}{\vec{\nabla}}
\newcommand{\delx}{\partial_x}
\newcommand{\dely}{\partial_y}
\newcommand{\delz}{\partial_z}


\begin{document}
% pagestyle nicht global festgelegt, da sonst bei allen Seiten Überschrift ist
% daher muss hier fancy aktiviert werden (für eine Seite, daher thispagestyle)
\thispagestyle{fancy}

\section*{Aufgabe 1}
% ...
\mbox{}
\par{a)} $ \quad \nabvec(x^2 + xz - z^2 + 3xyz)= 
\begin{pmatrix}
	\delx \\
	\dely \\
	\delz 
\end{pmatrix}
(x^2 + xz - z^2 + 3xyz)
	= 
\begin{pmatrix}
	\delx \ (x^2 + zx + 3xyz) \\
	\dely \ 3xyz \\
	\delz \ xz - z^2 + 3xyz
\end{pmatrix}
	=
\begin{pmatrix}
	2x + 3yz + z \\
	3xz \\
	x - 2z + 3xy
\end{pmatrix}
	$
\vspace{0.5cm} \par{b)} 
$ \nabvec \cdot
\begin{pmatrix}
	2xz+8x \\ e^x + y(\text{sin}^2(xyz) + \text{cos}(xy)) \\ \text{ln}(y^4) + 4xy + 7z^3
\end{pmatrix}
=
\begin{pmatrix}
	2z + 8 \\
	\sin^2(xyz) + cos(xy) + xyz*\sin(2*xyz) + xy*\sin(xy) \\
	21 z^2
\end{pmatrix}
$
\vspace{0.5cm} \par{c)}
\section*{Aufgabe 2}
% ...
\begin{align*}
	\text{a)} \quad \nabvec \cdot \left(\nabvec \times \vec A \right) = 
\begin{pmatrix}	\delx \\ \dely \\ \delz \end{pmatrix} \cdot
\begin{pmatrix} 
	\dely A_z - \delz A_y \\ \delz A_x - \delx A_z \\ \delx A_y - \dely A_x
\end{pmatrix}
	&=
\delx\dely A_z - \delx\delz A_y + \dely\delz A_x - \dely\delx A_z + \delz\delx A_y - \delz\dely A_x \\
	\text{(Satz von Schwarz)}\Rightarrow &=
\delx\dely A_z - \delx\delz A_y + \dely\delz A_x - \delx\dely A_z + \delx\delz A_y - \dely\delz A_x \\
	&= 0
\end{align*}
\begin{align*}
	\text{b)} 
\end{align*}
\section*{Aufgabe 3}
% ...
\section*{Aufgabe 4}
% ...
\section*{Aufgabe 5}

\end{document}
