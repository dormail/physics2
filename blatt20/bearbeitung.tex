
\documentclass[11pt a4paper]{article}
\usepackage[margin=2cm]{geometry}
\usepackage{amsmath, amssymb}
\usepackage{graphicx}
\usepackage{float}
\usepackage{aligned-overset}
\usepackage{subcaption}
\usepackage{tabularx} % fuer gleichungen nebeneinander

% partielle ableitungen
\newcommand{\delr}{\partial_r}
\newcommand{\deltheta}{\partial_\theta}
\newcommand{\delphi}{\partial_\varphi}

% elektrische feldkonstante
\newcommand{\epsz}{\epsilon_0}
% 1 / 4pi eps
\newcommand{\kco}{\frac{1}{4\pi\epsilon_0}}

% rotation
\newcommand{\rot}{\text{rot}}

% fancy header
\usepackage{fancyhdr}
\fancyhf{}
% vspaces in den headern fuer Distanzen notwendig
% linke Seite: Namen der Abgabegruppe
\lhead{\textbf{Matthias Maile\\Roman Surma}\vspace{1.5cm}}
% rechte Seite: Modul, Gruppe, Semester
\rhead{\textbf{Physik II - Gruppe 2\\Sommersemester 2020}\vspace{1.5cm}}
% Center: nr. des blattes
\chead{\vspace{2.5cm}\huge{\textbf{20. Übungsblatt}}}
% benoetigt damit der eigentliche Text nicht in der Überschrift steckt
\setlength{\headheight}{4cm}

% zum zeichnen tikz
\usepackage{tikz}

% fuer fabigen text
\usepackage{xcolor}

% irgendwas mit figures
\usepackage{subcaption}

\begin{document}
\thispagestyle{fancy}
\section{Aufgabe 1}

\newpage
\setlength{\headheight}{0cm}

\section*{Aufgabe 3}

Gesamtimpedanze der Schaltung:
\begin{align*}
	\frac1{Z_\text{ges}}
	&= \frac1{Z_\text{R}} + \frac1{Z_\text{C}} + \frac1{Z_\text{L}} \\
	&= \frac1R + i\omega C + \frac1{i\omega L}
\end{align*}

\newpage

\section*{Aufgabe 5}

\quad (a) Bei der Schaltung handelt es sich um einen Tiefpass. \\
Bei niedrigen Frequenzen verhält sich die Induktivität 
wie ein normales Kabel ($Z_L \approx 0$), erst bei hohen Frequenzen muss der Strom mehr gegen die 
Selbstinduktivität ``ankämpfen``.
\[ V(\omega) = \frac{R_a}{Z_\text{ges}} = \frac{R_a}{R + R_a + i\omega L} \]
\\

(b) Bei der Schaltung handelt es sich um einen Hochpass. \\
Bei niedrigen Frequenzen nähert sich das Verhalten der
Schaltung der bei Gleichstrom an, wodurch die Impedanz des Kondensators gegen unendlich geht.
\[ V(\omega) = \frac{R_a}{Z_\text{ges}} = \frac{R_a}{R + R_a + \frac{1}{i\omega c}} \]
\\

(c) Die Schaltung ist ein Hochpass. \\
Bei niedrigen Frequenzen würde der meiste Strom nicht durch den Verbraucher, sondern durch die Induktivität laufen.
Erst bei höheren Frequenzen steigt die Impedanz der Induktivität und mehr Strom fließt durch den Verbraucher.
\[
	Z_a = \frac{R_a \ Z_L}{R_a + Z_L} = \frac{R_a \ i\omega L}{R_a + i\omega L} 
	\Rightarrow 
	V(\omega) = \frac{Z_a}{Z_\text{ges}} 
	= \frac{R_a \ i\omega L}{R_a + i\omega L} \frac{1}{R + \frac{R_a \ i\omega L}{R_a + i\omega L}}
	= \frac{\omega L - iR_a}{R R_a \omega L + \omega L - iR_a}
\]
\\

(d) Die Schaltung ist ein Tiefpass. \\
Das Verhalten ist das Umgekehrte zu der bei (c). Hier ist es jedoch so, dass bei niedrigen Frequenzen die Impedanz
des Kondensators gegen unendlich geht und bei hohen Frequenzen durchlässig wird. Dadurch fließt bei tiefen
Frequenzen viel Strom durch den Verbraucher, bei hohen wenig.
\[
	Z_a = \frac{R_a}{1 + R_a i\omega C}
	\Rightarrow 
	V(\omega) = \frac{Z_a}{Z_\text{ges}} 
	= \frac{R_a}{1 + R_a i\omega C} \frac{1}{R + \frac{R_a}{1 + R_a i\omega C}}
	= \frac{R_a}{R + R_a + RR_a i\omega C}
\]
\\

(e) Die Schaltung ist ein Bandpass. \\
Bei hohen Frequenzen ``scheitert`` der Strom an der Induktivität, bei niedrigen an dem Kondensator.
\[
	V(\omega) = \frac{Z_a}{Z_\text{ges}} = \frac{R_a}{R + i\omega L + \frac{1}{i \omega C} + R_a}
\]
\\

(f) Die Schaltung ist ein Sperrfilter. \\
Sehr hohe Frequenzen können einfach durch den Kondensator, niedrige durch die Induktivität. Frequenzen in einem
mittleren Frequenzbereich werden jedoch nicht stark durchgelassen.
\[
	V(\omega) = \frac{Z_a}{Z_\text{ges}} 
	= \frac{R_a}{R + \frac{L}{C} \frac{i\omega C}{1 - \omega^2 LC} + R_a}
\]




\end{document}
