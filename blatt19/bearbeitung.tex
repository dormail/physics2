\documentclass[11pt a4paper]{article}
\usepackage[margin=2cm]{geometry}
\usepackage{amsmath, amssymb}
\usepackage{graphicx}
\usepackage{float}
\usepackage{aligned-overset}
\usepackage{subcaption}

% partielle ableitungen
\newcommand{\delr}{\partial_r}
\newcommand{\deltheta}{\partial_\theta}
\newcommand{\delphi}{\partial_\varphi}

% elektrische feldkonstante
\newcommand{\epsz}{\epsilon_0}
% 1 / 4pi eps
\newcommand{\kco}{\frac{1}{4\pi\epsilon_0}}

% rotation
\newcommand{\rot}{\text{rot}}

% fancy header
\usepackage{fancyhdr}
\fancyhf{}
% vspaces in den headern fuer Distanzen notwendig
% linke Seite: Namen der Abgabegruppe
\lhead{\textbf{Matthias Maile\\Roman Surma}\vspace{1.5cm}}
% rechte Seite: Modul, Gruppe, Semester
\rhead{\textbf{Physik II - Gruppe 2\\Sommersemester 2020}\vspace{1.5cm}}
% Center: nr. des blattes
\chead{\vspace{2.5cm}\huge{\textbf{19. Übungsblatt}}}
% benoetigt damit der eigentliche Text nicht in der Überschrift steckt
\setlength{\headheight}{4cm}

% zum zeichnen tikz
\usepackage{tikz}

% fuer fabigen text
\usepackage{xcolor}

% irgendwas mit figures
\usepackage{subcaption}

\begin{document}
\thispagestyle{fancy}
\section*{Aufgabe 1}

\newpage
\setlength{\headheight}{0cm}

\section*{Aufgabe 2}
a)
\begin{enumerate}
	\item[(i)] 
		$ \epsilon_{ijk} 
		= \alpha \ \epsilon_{jki} 
		= \alpha \ \epsilon_{ijk}
		\Rightarrow \alpha = 1$
	\item[(ii)]
		$ \epsilon_{ijk} 
		= \alpha \ \epsilon_{jik} 
		= -\alpha \ \epsilon_{ijk}
		\Rightarrow \alpha = -1$
	\item[(iii)]
		$ \epsilon_{iik} 
		= \alpha \ \epsilon_{ijk} 
		= 0
		\Rightarrow \alpha = 0$
\end{enumerate}
\vspace{0.5cm}
b) Zur besseren Lesbarkeit wird Einsteinsche Summenkonvention verwendet.
\begin{enumerate}
	\item[(i)]
		$ \epsilon_{ijk} \ \epsilon_{ijk} = \epsilon_{ijk}^2
		= \begin{cases} 
			1 & \text{für Permutationen von 1,2,3} \\
			0 & \text{sonst}
		\end{cases} $
	\item[(ii)]
		$ \epsilon_{ijk} \ \epsilon_{kmn} =
		\begin{cases}
			1 & \text{für }  i=m \wedge j=n \\
			-1 & \text{für } i=n \wedge j=m \\
			0 & \text{sonst} 
		\end{cases} $
\end{enumerate}
\vspace{0.5cm}
c)
\begin{align*}
	\vec a \times (\vec b \times \vec c)
	% aeusseres kreuzprodukt in indize schreibweise
	&= \epsilon_{ijk} \ \hat i \cdot a_j \cdot (\vec b \times \vec c)_k
	\\
	% inneres kreuzprodukt in indizesnotation
	&= \epsilon_{ijk} \ \hat i \cdot a_j \cdot 
	\epsilon_{kmn} \cdot b_m \cdot c_n
	\\
	% ergebnis aus b) (ii) einsetzen
	b)(ii) \Rightarrow
	&= \hat i \cdot b_i a_j c_j - \hat i \cdot c_i a_j b_j
	\\
	% \hat i und b_i bzw c_i zu vektoren umformen
	&= \vec b \cdot a_j c_j - \vec c \cdot  a_j b_j
	\\
	% a c skalarprodukt erzeugen
	&= \vec b \ (\vec a \cdot \vec c) - \vec c \ (\vec a \cdot \vec b)
	\\
\end{align*}

\newpage
\section*{Aufgabe 3}
\newpage
\section*{Aufgabe 4}
\newpage
\section*{Aufgabe 5}
\newpage

\end{document}
