\documentclass[11pt a4paper]{article}
\usepackage[margin=2cm]{geometry}
\usepackage{amsmath, amssymb}
\usepackage{graphicx}
\usepackage{float}
\usepackage{aligned-overset}

% partielle ableitungen
\newcommand{\delr}{\partial_r}
\newcommand{\deltheta}{\partial_\theta}
\newcommand{\delphi}{\partial_\varphi}

% elektrische feldkonstante
\newcommand{\epsz}{\epsilon_0}
% 1 / 4pi eps
\newcommand{\kco}{\frac{1}{4\pi\epsilon_0}}

% fancy header
\usepackage{fancyhdr}
\fancyhf{}
% vspaces in den headern fuer Distanzen notwendig
% linke Seite: Namen der Abgabegruppe
\lhead{\textbf{Etem Kalyon\\Matthias Maile\\Roman Surma}\vspace{1.5cm}}
% rechte Seite: Modul, Gruppe, Semester
\rhead{\textbf{Physik II - Gruppe 2\\Sommersemester 2020}\vspace{1.5cm}}
% Center: nr. des blattes
\chead{\vspace{2.5cm}\huge{\textbf{16. Übungsblatt}}}
% benoetigt damit der eigentliche Text nicht in der Überschrift steckt
\setlength{\headheight}{4cm}

% zum zeichnen tikz
\usepackage{tikz}

\begin{document}
\thispagestyle{fancy}
\section*{Aufgabe 1: Eine geladene Linie}
\par{a)}
Die drei dimensionale Linienladungsdichte $\rho$ lautet:
\[
	\rho(\vec r) =
	\sigma \cdot \delta(x) \cdot \delta(y) \cdot \Theta\left( z + \frac a 2 \right)
	\cdot \Theta\left(\frac a 2 - z\right)
	\qquad \text{ mit } \sigma \text{ als Linienladungsdichte }
	\sigma = \frac Q a
\]


\newpage
\setlength{\headheight}{0cm}
\section*{Aufgabe 3: Spiegelladungen}
Zum Erfüllen der Randbedingung setzen 3 weitere Ladungen in das System,
die gegenüberliegende Ladung ist gleichnamig. Die Ladungen im 2. und im 4.
Quadranten sind jedoch negativ geladen. Die Anordnung sieht dann wie folgt
aus:
\newline
\begin{center}
\begin{tikzpicture}
	% die x und y achse
	\draw[gray, thick, ->] (0, -3) -- (0, 3);
	\draw[gray, thick, ->] (-5, 0) -- (5,0);
	% die Ladungen mit annotation
	\filldraw [black] (3,2) circle (2pt);
	\draw (3,2) node [right] {$Q$};
	\filldraw [black] (-3,2) circle (2pt);
	\draw (-3,2) node [left] {$-q$};
	\filldraw [black] (-3,-2) circle (2pt);
	\draw (-3,-2) node [left] {$q$};
	\filldraw [black] (3,-2) circle (2pt);
	\draw (3,-2) node [right] {$-q$};
	% weg zu Ladung q0
	\draw[thick, dashed] (3,0) -- (3,2);
	\draw[thick, dashed] (0,2) -- (3,2);
	% anotation an der achse
	\draw (0,2) node [left] {$b$};
	\draw (3,0) node [below] {$a$};
\end{tikzpicture}
\end{center}
Durch die Multipolentwicklung können wir das elektrische Potential im Raum ermitteln:
\begin{align*}
	\phi (\vec r)
	&= \frac1{4\pi\epsz} \sum_i \frac{q_i}{d(\vec r, \vec{q_i})} \\
	% ladungen und distanzen einsetzen
	&= \kco \left[
		\underbrace{\frac{Q}{\sqrt{(x-a)^2 + (y-b)^2}}}_{
			\text{Durch Ladung } Q}
		- \underbrace{\frac{Q}{\sqrt{(x+a)^2 + (y-b)^2}} }_{
			\text{Spiegelladung im 2. Quadrant}}
		+ \underbrace{\frac{Q}{\sqrt{(x + a)^2 + (y+b)^2}} }_{
			\text{Spiegelladung im 3. Quadrant}}
		- \underbrace{\frac{Q}{\sqrt{(x-a)^2 + (y+b)^2}} }_{
			\text{Spiegelladung im 4. Quadrant}}
	\right]
\end{align*}
Die Randbedingung $\phi = 0$ auf den Platten ist damit auch erfüllt:
\begin{align*}
	\phi \left( \begin{pmatrix} x \\ 0 \end{pmatrix} \right) 
	&= \kco \left[
		\frac{Q}{\sqrt{(x-a)^2 + (0-b)^2}}
		- \frac{Q}{\sqrt{(x+a)^2 + (0-b)^2}}
		+ \frac{Q}{\sqrt{(x + a)^2 + (0+b)^2}}
		- \frac{Q}{\sqrt{(x-a)^2 + (0+b)^2}}
	\right] \\
	% zusammen fassen, 0 entfernen
	&= \kco \left[
		\frac{Q}{\sqrt{(x-a)^2 + b^2}}
		- \frac{Q}{\sqrt{(x+a)^2 + b^2}}
		+ \frac{Q}{\sqrt{(x + a)^2 + b^2}}
		- \frac{Q}{\sqrt{(x-a)^2 + b^2}}
	\right] \\
	&= 0 \\
	% jetzt noch mal alles für y achse
	\phi \left( \begin{pmatrix} 0 \\ y \end{pmatrix} \right) 
	&= \kco \left[
		\frac{Q}{\sqrt{(0-a)^2 + (y-b)^2}}
		- \frac{Q}{\sqrt{(0+a)^2 + (y-b)^2}}
		+ \frac{Q}{\sqrt{(0 + a)^2 + (y+b)^2}}
		- \frac{Q}{\sqrt{(0-a)^2 + (y+b)^2}}
	\right] \\
	% 0 entfernen
	&= \kco \left[
		\frac{Q}{\sqrt{a^2 + (y-b)^2}}
		- \frac{Q}{\sqrt{a^2 + (y-b)^2}}
		+ \frac{Q}{\sqrt{a^2 + (y+b)^2}}
		- \frac{Q}{\sqrt{a^2 + (y+b)^2}}
	\right] \\
	&= 0
\end{align*}


\end{document}
