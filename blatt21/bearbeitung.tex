\documentclass[11pt a4paper]{article}
\usepackage[margin=2cm]{geometry}
\usepackage{amsmath, amssymb}
\usepackage{graphicx}
\usepackage{float}
\usepackage{aligned-overset}
\usepackage{subcaption}
\usepackage{tabularx} % fuer gleichungen nebeneinander
\usepackage{wrapfig} % damit figures neben text sien koennen

% partielle ableitungen
\newcommand{\delr}{\partial_r}
\newcommand{\delt}{\partial_t}
\newcommand{\deltheta}{\partial_\theta}
\newcommand{\delphi}{\partial_\varphi}

% elektrische feldkonstante
\newcommand{\epsz}{\epsilon_0}
% 1 / 4pi eps
\newcommand{\kco}{\frac{1}{4\pi\epsilon_0}}

% div und rot
\newcommand{\diver}{\vec \nabla \cdot}
\newcommand{\rot}{\vec \nabla \times}

% hyperbolische funktionen
\newcommand{\arsinh}{\text{arsinh}}
\newcommand{\arcosh}{\text{arcosh}}
\newcommand{\artanh}{\text{artanh}}

% fuer impedanzen 
\newcommand{\omegaC}{\omega C}
\newcommand{\omegaL}{\omega L}
\newcommand{\omegaR}{\omega R}
\newcommand{\omegaLC}{\omega LC}
\newcommand{\omegaLCR}{\omega LCR}
\newcommand{\omegaCR}{\omega CR}

% fancy header
\usepackage{fancyhdr}
\fancyhf{}
% vspaces in den headern fuer Distanzen notwendig
% linke Seite: Namen der Abgabegruppe
\lhead{\textbf{Matthias Maile\\Roman Surma}\vspace{1.5cm}}
% rechte Seite: Modul, Gruppe, Semester
\rhead{\textbf{Physik II - Gruppe 2\\Sommersemester 2020}\vspace{1.5cm}}
% Center: nr. des blattes
\chead{\vspace{2.5cm}\huge{\textbf{21. Übungsblatt}}}
% benoetigt damit der eigentliche Text nicht in der Überschrift steckt
\setlength{\headheight}{4cm}

% zum zeichnen tikz
\usepackage{tikz}

% fuer fabigen text
\usepackage{xcolor}

% irgendwas mit figures
\usepackage{subcaption}

\begin{document}
\thispagestyle{fancy}

\section*{Aufgabe 1}
a) 
\begin{align*}
	Z_{AB}
	&= Z_C + Z_{LR} \\
	&= \frac{i\omega C} + \frac{i\omega LR}{i \omega L + R} \\
	&= \frac{i \omega L + R + i \omega C (i \omega LR)}{i \omegaC (i\omegaL + R)} \\
	&= \frac{R - \omega^2 LCR + i \omega L}{i\omega CR - \omega^2 LC}
\end{align*}
Für die relle Impedanz:
\begin{align*}
	\text{Im} \left( Z_{AB} \right)
	&= \frac{\omega L (- \omega^2 LC) - (R - \omega^2 LCR) \cdot \omegaCR}{(\omega CR)^2 + (\omega^2 LC)^2} 
	\overset{!}{=} 0 \\
	% bruch entfernen
	\Rightarrow
	0
	&= \omega L (- \omega^2 LC) - (R - \omega^2 LCR) \cdot \omegaCR \\
	% vereinfachen
	&= -\omega^3 L^2 C - \omega CR^2 + \omega^3 LC^2R^2 \\
	% durch omega teilen
	\Rightarrow
	0
	&= -\omega^2 L^2 C - CR^2 + \omega^2 LC^2R^2 \\
	\Leftrightarrow
	CR^2
	&= -\omega^2 L^2 C + \omega^2 LC^2R^2 \\
	% nach omega^2 umstellen
	\Rightarrow
	\omega^2
	&= \frac{CR^2}{L^2C^2R^2 - L^2C} \\
	\Rightarrow
	\omega
	&= \sqrt{\frac{R^2}{L^2CR^2 - L^2}}
\end{align*}
b) Die an der LCR-Schaltung abfallende Spannung $U_{AB}$ ist bestimmt durch
\[ U_{AB} = \frac{Z_{AB}}{R_i + Z_{AB}} U_0 \]
Dann lautet die elektrische Leistung der LCR-Schaltung:
\begin{align*}
	P
	&= \frac{\Delta U^2}{Z_{AB}} \\
	&= \frac{Z_{AB}^2}{(R_i+Z_{AB})^2} \frac{1}{Z_{AB}} U_0 \\
	&= \frac{Z_{AB}}{(R_i+Z_{AB})^2} U_0 
\end{align*}
Diese wird maximal für
\begin{align*}
	d_{Z_{AB}} P
	&= U \frac{(R_i+Z_{AB})^2 - Z_{AB} \cdot 2 \cdot (R_i + Z_{AB})}{(R_i+Z_{AB})^4} \overset != 0 \\
	\Leftrightarrow
	0
	&= R_i + Z_{AB} - 2 Z_{AB} \\
	\Rightarrow
	Z_{AB} &= R_i
\end{align*}

\newpage
\setlength{\headheight}{0cm}

c) 
\begin{align*}
	R_i
	&= \frac{R - \omega^2 LCR + i \omega L}{i\omega CR - \omega^2 LC} \\
	% quotient berechnen
	Z_{AB} \in \mathbb{R} \Rightarrow
	&= \frac{(R - \omega^2 LCR) (-\omega^2 LC) + \omega^2 LCR}{(\omega CR)^2 + (\omega^2 LC)^2} \\
	&= \frac{\omega^4 L^2 C^2 R - \omega^2 LCR + \omega^2 LCR}{(\omega CR)^2 + (\omega^2 LC)^2} \\
	&= \frac{\omega^4 L^2 C^2 R}{(\omega CR)^2 + (\omega^2 LC)^2} \\
	&= \frac{\omega^2 L^2 R}{R^2 + \omega^2 L^2} \\
	% nach 0 umstellen
	\Rightarrow
	0
	&= \frac{\omega^2 L^2 R}{R^2 + \omega^2 L^2} - R_i \\
	% bruch entfernen
	\Leftrightarrow
	0
	&= \omega^2 L^2 R - R_iR^2 - R_i\omega^2L^2 \\
	% fuer pq formel
	\Leftrightarrow
	0
	&= R^2 - \frac{\omega^2L^2}{R_i} R + \omega^2 L^2
\end{align*}
Mit der pq-Formel erhalten wir eine Lösung:
\[
	R = \frac{\omega^2L^2}{2 R_i} \pm \sqrt{\left( \frac{\omega^2L^2}{2 R_i}\right)^2 - \omega^2 L^2} 
\]


\newpage
\setlength{\headheight}{0cm}

\section*{Aufgabe 2}
Die Leistung des Heizlüfters hängt vom Spannungsabfall an diesem ab:
\[ P_H = \frac{\Delta U^2}{R_2} \]
Der Spannungsabfall folgt aus den Kirchhoffschen Regeln:
\[ \Delta U = U_H \frac{R_2}{R_1 + R_2} \]
Dann können wir den benötigten Widerstand genau berechnen:
\begin{align*}
	P_H 
	&= U_H^2 \frac{R_2^2}{(R_1 + R_2)^2} \frac{1}{R_2} \\
	&= U_H^2 \frac{R_2}{(R_1 + R_2)^2} \\
	% 1. umformung
	\Leftrightarrow
	P_H (R_1 + R_2)^2 
	&= U_H^2 R_2 \\
	% binomische formel, nach 0 umstellen
	\Leftrightarrow
	0 
	&= P_H R_1^2 + 2 P_H R_1 R_2 + P_H R_2^2 - U_H^2 R_2 \\
	\Leftrightarrow
	0
	&= R_2^2 + \left( 2 R_1 - \frac{U_H^2}{P_H} \right) R_2 + R_1^2 \\
\end{align*}

\newpage

\section*{Aufgabe 3}
a) Da die Leitung einen Widerstand besitzt, verrichtet der Strom an diesem Arbeit durch Aufheizen des Kabels.
\[ P = U \cdot I = I^2 \cdot R \]
Dieser ist proportional zum Quadrat der Stromstärke $I$, d.h. diese sollte minimiert werden. Dafür muss die 
Spannung $U$ angehoben werden.
b)
\[
	P = I^2 \cdot R = 500^2 \ A^2 \cdot 0.2 \ k\Omega = 5 \cdot 10^4 \ W
\]

\newpage

\section*{Aufgabe 5}
a) 
\begin{enumerate}
	\item Mit Gaußschem Satz
	\[ 
		\diver \vec E = \frac{\rho}{\epsz} 
		\Rightarrow
		\int_V \diver \vec E \ dV = \int_V \frac{\rho}{\epsz} \ dV = \frac{q_\text{enq}}{\epsz}
		\Rightarrow
		\oint_{\partial_V} \vec E \ d\vec A = \frac{q_\text{enq}}{\epsz}
	\]
	\item Ebenfalls mit Gaußschem Satz
	\[
		\diver \vec B = 0
		\Rightarrow
		\int_V \diver \vec B \ dV = 0
		\Rightarrow
		\oint_{\partial_V} \vec B \ d\vec A = 0
	\]
	\item Mit Satz von Stokes
	\[
		\rot \vec E = -\delt \vec B
		\Rightarrow
		\int_A \rot \vec E \ d\vec A = \int_A -\delt \vec B \ d\vec A
		\Rightarrow
		\oint_{\partial_A} \vec E \ d\vec r = -\delt \int_A \vec B \ \vec d\vec A = -\delt \phi
	\]
	\item 
\end{enumerate}

\end{document}
