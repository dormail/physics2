\documentclass[11pt a4paper]{article}
\usepackage[margin=2cm]{geometry}
\usepackage{amsmath, amssymb}
\usepackage{graphicx}
\usepackage{float}
\usepackage{aligned-overset}
\usepackage{subcaption}
\usepackage{tabularx} % fuer gleichungen nebeneinander
\usepackage{wrapfig} % damit figures neben text sien koennen

% partielle ableitungen
\newcommand{\delr}{\partial_r}
\newcommand{\delt}{\partial_t}
\newcommand{\deltheta}{\partial_\theta}
\newcommand{\delphi}{\partial_\varphi}

% elektrische feldkonstante
\newcommand{\epsz}{\epsilon_0}
% 1 / 4pi eps
\newcommand{\kco}{\frac{1}{4\pi\epsilon_0}}

% div und rot
\newcommand{\diver}{\vec \nabla \cdot}
\newcommand{\rot}{\vec \nabla \times}

% hyperbolische funktionen
\newcommand{\arsinh}{\text{arsinh}}
\newcommand{\arcosh}{\text{arcosh}}
\newcommand{\artanh}{\text{artanh}}

% fancy header
\usepackage{fancyhdr}
\fancyhf{}
% vspaces in den headern fuer Distanzen notwendig
% linke Seite: Namen der Abgabegruppe
\lhead{\textbf{Matthias Maile\\Roman Surma}\vspace{1.5cm}}
% rechte Seite: Modul, Gruppe, Semester
\rhead{\textbf{Physik II - Gruppe 2\\Sommersemester 2020}\vspace{1.5cm}}
% Center: nr. des blattes
\chead{\vspace{2.5cm}\huge{\textbf{21. Übungsblatt}}}
% benoetigt damit der eigentliche Text nicht in der Überschrift steckt
\setlength{\headheight}{4cm}

% zum zeichnen tikz
\usepackage{tikz}

% fuer fabigen text
\usepackage{xcolor}

% irgendwas mit figures
\usepackage{subcaption}

\begin{document}
\thispagestyle{fancy}

\section*{Aufgabe 1}

\newpage
\setlength{\headheight}{0cm}

\section*{Aufgabe 3}
a) Da die Leitung einen Widerstand besitzt, verrichtet der Strom an diesem Arbeit durch Aufheizen des Kabels.
\[ P = U \cdot I = I^2 \cdot R \]
Dieser ist proportional zum Quadrat der Stromstärke $I$, d.h. diese sollte minimiert werden. Dafür muss die 
Spannung $U$ angehoben werden.
b)
\[
	P = I^2 \cdot R = 500^2 \ A^2 \cdot 0.2 \ k\Omega = 5 \cdot 10^4 \ W
\]

\newpage

\section*{Aufgabe 5}
a) 
\begin{enumerate}
	\item Mit Gaußschem Satz
	\[ 
		\diver \vec E = \frac{\rho}{\epsz} 
		\Rightarrow
		\int_V \diver \vec E \ dV = \int_V \frac{\rho}{\epsz} \ dV = \frac{q_\text{enq}}{\epsz}
		\Rightarrow
		\oint_{\partial_V} \vec E \ d\vec A = \frac{q_\text{enq}}{\epsz}
	\]
	\item Ebenfalls mit Gaußschem Satz
	\[
		\diver \vec B = 0
		\Rightarrow
		\int_V \diver \vec B \ dV = 0
		\Rightarrow
		\oint_{\partial_V} \vec B \ d\vec A = 0
	\]
	\item Mit Satz von Stokes
	\[
		\rot \vec E = -\delt \vec B
		\Rightarrow
		\int_A \rot \vec E \ d\vec A = \int_A -\delt \vec B \ d\vec A
		\Rightarrow
		\oint_{\partial_A} \vec E \ d\vec r = -\delt \int_A \vec B \ \vec d\vec A = -\delt \phi
	\]
	\item 
\end{enumerate}

\end{document}
