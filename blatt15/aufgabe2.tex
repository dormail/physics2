\documentclass{article}
\usepackage[margin=2cm]{geometry}
\usepackage{amsmath, amssymb}

% partielle ableitungen
\newcommand{\delr}{\partial_r}
\newcommand{\deltheta}{\partial_\theta}
\newcommand{\delphi}{\partial_\varphi}

% elektrische feldkonstante
\newcommand{\epsz}{\epsilon_0}

\begin{document}
\section*{Aufgabe 2}
\paragraph{a)}
$\Delta$ in Kugelkoordinaten:
\[
	\Delta f = \frac 1 {r^2}(
	\delr ( r^2 \delr f) +
	\frac 1 {\sin\theta} \deltheta (\sin\theta \deltheta f) +
	\frac 1 {\sin^2\theta} \delphi^2f)
\]
Die Poisson-Gleichung lautet:
\[
	\Delta \phi(r,\theta,\varphi) = \frac{-\rho(r,\theta,\varphi)}
	{\epsilon_0}
\]
Da die Ladung als Kugel (also symmetrisch zu $\theta$ und $\varphi$) 
angeordnet ist, vereinfacht sich diese:
\[
	\Delta \phi(r) = \frac{-\rho(r)}{\epsilon_0}
\]
Die partiellen Ableitungen zu $\theta$ und $\varphi$ sind dem ensprechend 
auch 0. \\
Durch zweifaches Integrieren können wir für $r > R$ ermitteln:
\begin{align*}
	\Delta \phi = 
	\frac 1 {r^2} \delr \left(r^2 \delr \phi \right) 
	&= \frac{- \rho_0}{\epsz} = 0\\
	\Leftrightarrow
	\delr \left( r^2 \delr \phi \right) &= 0 \\
	\Leftrightarrow
	r^2 \delr \phi  &= \alpha \\
	\Leftrightarrow
	\delr \phi  &= \frac{\alpha}{r^2} \\
	\Leftrightarrow
	\phi  &= -\frac{\alpha}{r} + \beta\\
\end{align*}
Ebenso kann man für $r \leq R$ ermitteln:
\begin{align*}
	\Delta \phi = 
	\frac 1 {r^2} \delr \left(r^2 \delr \phi \right) 
	&= \frac{- \rho_0}{\epsz} \\
	\Leftrightarrow
	\delr \left( r^2 \delr \phi \right) 
	&= \frac{- \rho_0 }{\epsz} r^2  \\
	\Leftrightarrow
	r^2 \delr \phi
	&= \frac{- \rho_0 }{3  \epsz} r^3 + \gamma \\
	\Leftrightarrow
	\delr \phi
	&= \frac{- \rho_0 }{3  \epsz} r + \frac \gamma {r^2} \\
	\Leftrightarrow
	\phi
	&= \frac{- \rho_0 }{6  \epsz} r^2 - \frac \gamma r + \delta \\
\end{align*}
\paragraph{b)}
\par{1.}
Für die erste RB $-\infty < \lim_{r\rightarrow 0} \phi(r) < \infty$ muss 
gelten:
\[
	\lim_{r\rightarrow 0} \phi(r) = \lim_{r\rightarrow 0} 
	\frac{- \rho_0 }{6  \epsz} r^3 - \frac \gamma r + \delta
	\overset{!}{\in} (-\infty, \infty)
	\Rightarrow
	\gamma = 0
\]
\par{2.}
Die Bedingung, dass das Potential für $r \rightarrow \infty$ gegen 0 gehen
soll bewirkt, dass die Werte des Potentials auf einen festen Wert gesetzt werden. \\
Ohne diese könnte man durch $\beta$ und $\delta$ das Potential beliebig nach
oben und unten Verschieben. Einen Punkt im Unendlichen als Referenz zu 
wählen, ist jedoch die Konvention in der Physik.
\[
	\lim_{r \rightarrow \infty} \phi(r) = \lim_{r \rightarrow \infty} 
	-\frac{\alpha}{r} + \beta = \beta \overset{!}{=} 0
	\Rightarrow \beta = 0
\]
Damit folgt das Potential $\phi(r)$ und die Ableitung $\phi^\prime(r)$:
\[
	\phi(r) =
	\begin{cases}
		-\frac \alpha r & \text{falls } r > R \\
		-\frac{\rho_0}{6\epsz} r^2 + \delta 
		& \text{falls } r \leq R
	\end{cases}
	\qquad
	\phi^\prime(r) = 
	\begin{cases}
		\frac\alpha{r^2} 
		& \text{falls } r > R \\
		-\frac{\rho_0}{3\epsz} r
		& \text{falls } r \leq R
	\end{cases}
\]
\newpage
\par{3.}
	



\end{document}
