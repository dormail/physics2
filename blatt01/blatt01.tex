\documentclass[a4paper]{article}
\usepackage[utf8]{inputenc}
\usepackage{fancyhdr}
\usepackage[margin=2.5cm]{geometry}
\usepackage{amsmath, amssymb}
\usepackage{aligned-overset}


% eigene commands
\newcommand{\epszero}{\epsilon_0}
% partielle ableitungen in Kugelkoordinaten
\newcommand{\delr}{\partial_r}
\newcommand{\delp}{\partial_\phi}
\newcommand{\delt}{\partial_\theta}
% einheisvektoren in kugelkoordinaten
\newcommand{\er}{\hat{e}_r}
\newcommand{\et}{\hat{e}_\theta}
\newcommand{\ep}{\hat{e}_\phi}
% 1/4piepsilon0
\newcommand{\kco}{\frac{1}{4\pi\epsilon_0}}

\fancyhf{}
% vspaces in den headern fuer Distanzen notwendig
% linke Seite: Namen der Abgabegruppe
\lhead{\textbf{Etem Kalyon\\Matthias Maile\\Roman Surma}\vspace{1.5cm}}
% rechte Seite: Modul, Gruppe, Semester
\rhead{\textbf{Physik II - Gruppe 2\\Sommersemester 2020}\vspace{1.5cm}}
% Center: nr. des blattes
\chead{\vspace{2.5cm}\huge{\textbf{1. Übungsblatt}}}
% benoetigt damit der eigentliche Text nicht in der Überschrift steckt
\setlength{\headheight}{4cm} 


\begin{document}
% pagestyle nicht global festgelegt, da sonst bei allen Seiten Überschrift ist
% daher muss hier fancy aktiviert werden (für eine Seite, daher thispagestyle)
\thispagestyle{fancy}

\section*{Aufgabe 1}
% ...
\section*{Aufgabe 2}
\par{(a)} 
% flaechenladungsdichte der erdoberflaeche
Das Gaußsche Gesetz besagt:
\[ \int \vec{E} \  d\vec{A} = \frac{Q}{\epsilon_0} \]
Da $\vec{E}$ unabhängig vom Ort auf der Fläche A ist und parallel zum Normalenvektor liegt, kann der Betrag von $\vec{E}$ aus dem Integral gezogen werden.\\
$\vec{E}$ und $\hat{n}$ zeigen jedoch in verschiedene Richtungen; $\hat{n}$ von der Erde weg, das elektrische Feld richtung Erde. Daher entsteht ein negatives Vorzeichen.\\
Die Flächenladungsdichte $\sigma$ lautet somit:
\begin{align*}
	\int \vec E \ d\vec A &= -\vert \vec E \vert \int d\vec A = -\vert \vec E \vert * A = \frac{Q}{\epszero} \\
	\Rightarrow \sigma = \frac{Q}{A} &= -E*\epszero
	\approx -1.33 * 10^{-9} \frac{Q}{m^2}
\end{align*}

\par{(b)}
% gesamtladung
Wenn man die Ladungsdichte $\sigma$ aus (a) mit der Fläche $A$ multipliziert erhält man die Gesamtladung:
\[
	Q_{Gesamt} = \sigma * A = -EA\epszero = -\epszero * E * 4\pi r_E^2 
	\approx -6.77 * 10^5 C
\]

\par{(c)}
% fallende kugeln
Die Kraft die auf eine Kugel wirkt ist die Summe aus Gewichtskraft und Coulomkraft. Aus dem zweiten newtonschen Axiom folgt dann die Beschleunigung.
\[
	F_{Gesamt} = F_G + F_{Co} = mg + E*q_{Kugel} \Rightarrow a = g + \frac{E*q_{Kugel}}{m} 
\]
Aus der Formel für den freien Fall können wir die benötigten Zeiten für den Fall bestimmen:
\begin{align*}
	\Delta h &= \frac{1}{2} a*(\Delta t)^2 \\
	\Rightarrow \Delta t &= \sqrt{\frac{2*\Delta h}{a}}
\end{align*}
\[ 
	\Delta t_{ungeladen} = \sqrt{\frac{2*\Delta h}{g}} = \sqrt{\frac{2*2m}{10m/s^2}} \approx 0.6325s
\]
\[
	\Delta t_{geladen} = \sqrt{\frac{2*\Delta h}{g + \frac{E*q_{Kugel}}{m} }}
	= \sqrt{\frac{4m}{10m/s^2 + \frac{150N/C * 100 \mu C}{0.1 Kg} }}
	= \sqrt{\frac{4m}{10m/s^2 + 0.15 m/s^2}} \approx 0.6278s
\]
Die geladene Kugel erreicht die Erde ungefähr $0.0047s$ schneller als die Ungeladene.

\newpage
\setlength{\headheight}{0cm}
\section*{Aufgabe 3}
% ...
Die Coulomb- und die Gewichtskraft sind definiert als:
\[
	F_C = \kco \frac{q_1q_2}{r^2} \qquad F_G = m*g
\]
Wir definieren $q_3 = \frac{q_1 + q_2}{2} $, die Ladung  beider Kugeln nach dem Ladungsausgleich. Aus den trigonometrischen Identitäten folgt das Verhältniss von $F_C$ und $F_G$:
\begin{align*}
	\tan \theta &= \frac{F_C}{F_G} =  \kco \frac{q_1q_2}{mg \cdot r^2}
\end{align*}
Für $\theta_2$ lässt sich die das Verhältniss von $q_1$ und $q_2$ herleiten:
\begin{align*}
	\tan \theta_2 &= \kco \frac{q_3^2}{mg \cdot r_2^2} \\
	\Leftrightarrow
	q_3^2 &= \tan \theta_2 \cdot mg r_2^2 * 4\pi\epsilon_0 \\
	\Leftrightarrow
	q_3 = \frac{q_1 + q_2}{2} &= \sqrt{\tan \theta_2 \cdot mg r_2^2 * 4\pi\epsilon_0} \\
	\Leftrightarrow
	q_1 &= \sqrt{\tan \theta_2 \cdot mg r_2^2 * 16\pi\epsilon_0} - q_2 \\
\end{align*}
Aus $\theta_1$ folgt zuletzt:
\begin{align*}
	\tan \theta_1 &= \kco \frac{q_1q_2}{mg \cdot r_1^2} \\
	\tan \theta_1 &= \kco \frac{ \left( \sqrt{\tan \theta_2 \cdot mg r_2^2 * 16\pi\epsilon_0} - q_2 \right) q_2}{mg \cdot r_1^2} \\
	% umstellen fuer pq formel
	0 &= q_2^2 - q_2 \sqrt{\tan \theta_2 \cdot mg r_2^2 * 16\pi\epsilon_0}  + \tan \theta_1 * 4\pi\epsilon_0	*mg r_1^2 
\end{align*}


\section*{Aufgabe 4}
% ...
\par{a)}
E-Feld außerhalb einer geladenen Hohlkugel (Flächenladungsdichte $\sigma$):
\[ \vec E(r) = \kco \frac{Q_{Kugel}}{r^2} \hat r \]
Das Feld nah an der Kugel ($r = R_{Kugel}$) ist also:
\[
E(R) = \kco \frac{Q_{Kugel}}{R^2} = \kco \frac{\sigma * 4 \pi R^2}{R^2} = \frac \sigma {\epsilon_0}
\]
Damit die Durchschlagsfeldstärke nicht erreicht wird, darf $\sigma$ nicht höher als $2.656 * 10^{-5} \frac{C}{m^2}$ sein:
\[
\frac \sigma {\epsilon_0} < 3 * 10^6 \frac V m
\Leftrightarrow
\sigma < \epsilon_0 * 3 * 10^6 \frac V m = 2.656 * 10^{-5} \frac{C}{m^2}
\]
\vspace{0.5cm}

\par{b)}
Das E-Feld einer Linienladung mit Ladungsdichte $\lambda$, für $r \ll l$:
\[
	\vec E = \frac \lambda {2 \pi r \epsilon_0} \hat r
\]
Durch umstellen erhalten wir den Radius der ionisierten Luft (der Bereich wo $E \geq 3 * 10^6 \frac{V}{m} $:
\[
	r_{max} = \frac \lambda {E * 2 \pi \epsilon_0} = \frac{10^3 \frac C m}{3 * 10^6 \frac{V}{m} * 2 \pi \epsilon_0} \approx 5.99 m
	\] 

\newpage

\section*{Aufgabe 5}
% ...
\par{a)}
Nabla in Kugelkoordinaten:
\[ \nabla = \er \cdot \delr + \et \cdot \frac{1}{r} \delt + \ep \cdot \frac{1}{r \sin\theta} \delp \]
Die Divergenz eines Radialfeldes $\vec{E}(\vec{r}) = \alpha r^\beta \er$ lautet dadurch:
\begin{align*}
	\vec{\nabla} \cdot \vec{E}(\vec{r}) 
	&= \left(  \er \delr + \et \frac{1}{r} \delt + \ep  \frac{1}{r \sin\theta} \delp \right) \cdot \alpha r^\beta \er \\
	% skalar produkt aufgeloest
	&= \left(\er * \delr (\alpha r^\beta \er )
		+ \et * \frac{1}{r} \delt (\alpha r^\beta \er )
		+ \ep * \frac{1}{r\sin\theta} \delp (\alpha r^\beta \er) \right) \\
	% ableiten
	&= \left(\er \cdot \er \cdot \alpha \beta r^{\beta - 1}
		+ \frac{1}{r} \cdot \et \cdot \et \alpha r^\beta
		+ \frac{1}{r\sin\theta} \cdot \ep \cdot \ep \cdot \alpha r^\beta \right) \\
	% skalarprodukte
	&= \left(\alpha \beta r^{\beta - 1}
		+ \frac{1}{r} \alpha r^\beta
		+ \frac{1}{r\sin\theta}  \alpha r^\beta \right) \\
	% kuerzen
	&= \left( \alpha\beta r^{\beta - 1} +\alpha r^{\beta - 1} +\alpha r^{\beta - 1} \right) \\
	&= \alpha (\beta + 2) r^{\beta - 1}
\end{align*}
Da $\nabla \cdot \vec{E}(\vec{r}) = 1$ für alle $r$ erfüllt sein soll, muss die Gleichung unabhängig von $r$ sein:
\begin{align*}
	\alpha (\beta + 2) r^{\beta - 1} \overset{!}&{=} 1
	\quad \Rightarrow
	\beta = 1 \text{ damit Gleichung unabhängig von r ist}
	\\
	\Leftrightarrow
	\alpha * 3 &= 1 \\
	\Leftrightarrow
	\alpha &= \frac{1}{3} \\
	\Rightarrow \vec{E}(\vec{r}) &= \frac{1}{3} r \hat{e}_r
\end{align*}

\par{b)}
\begin{align*}
	V_{Kugel} 
	&= \iiint dV = \iiint \vec{\nabla} \cdot \vec{E}(\vec{r})dV \overset{\text{Satz v. Gauß}}{=}
	\iint \vec{E}(\vec{r})dA \\
	% einsetzen	
	&= \iint \frac{1}{3} R \hat{e}_r dA \\
	&= \int_0^{2\pi} \int_0^\pi \frac{R}{3} \ \hat{e}_r \cdot \hat{n} * R^2\sin\theta \ d\theta d\varphi \\
	% integral vereinfahen
	&= \frac{R^3}{3} \int_0^{2\pi} \int_0^\pi \sin\theta \ d\theta d\varphi \\
	% integral loesen
	&= \frac{R^2}{3} * 4\pi \\
	&= \frac{4}{3} * \pi * R^3
\end{align*}

\newpage
\section*{Aufgabe 6}
\[
	\vec F = \begin{pmatrix} y^3 \\ x^2 \\ z \end{pmatrix}
\]



\end{document}
