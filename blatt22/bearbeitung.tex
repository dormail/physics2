\documentclass[11pt a4paper]{article}
\usepackage[margin=2cm]{geometry}
\usepackage{amsmath, amssymb}
\usepackage{graphicx}
\usepackage{float}
\usepackage{aligned-overset}
\usepackage{subcaption}
\usepackage{tabularx} % fuer gleichungen nebeneinander
\usepackage{wrapfig} % damit figures neben text sien koennen

% partielle ableitungen
\newcommand{\delr}{\partial_r}
\newcommand{\delt}{\partial_t}
\newcommand{\deltheta}{\partial_\theta}
\newcommand{\delphi}{\partial_\varphi}

% elektrische feldkonstante
\newcommand{\epsz}{\epsilon_0}
% 1 / 4pi eps
\newcommand{\kco}{\frac{1}{4\pi\epsilon_0}}
% del fuer partial
\newcommand{\del}{\partial}

% div und rot
\newcommand{\diver}{\vec \nabla \cdot}
\newcommand{\rot}{\vec \nabla \times}

% hyperbolische funktionen
\newcommand{\arsinh}{\text{arsinh}}
\newcommand{\arcosh}{\text{arcosh}}
\newcommand{\artanh}{\text{artanh}}

% fuer impedanzen 
\newcommand{\omegaC}{\omega C}
\newcommand{\omegaL}{\omega L}
\newcommand{\omegaR}{\omega R}
\newcommand{\omegaLC}{\omega LC}
\newcommand{\omegaLCR}{\omega LCR}
\newcommand{\omegaCR}{\omega CR}

% fancy header
\usepackage{fancyhdr}
\fancyhf{}
% vspaces in den headern fuer Distanzen notwendig
% linke Seite: Namen der Abgabegruppe
\lhead{\textbf{Matthias Maile\\Roman Surma}\vspace{1.5cm}}
% rechte Seite: Modul, Gruppe, Semester
\rhead{\textbf{Physik II - Gruppe 2\\Sommersemester 2020}\vspace{1.5cm}}
% Center: nr. des blattes
\chead{\vspace{2.5cm}\huge{\textbf{22. Übungsblatt}}}
% benoetigt damit der eigentliche Text nicht in der Überschrift steckt
\setlength{\headheight}{4cm}

% zum zeichnen tikz
\usepackage{tikz}

% fuer fabigen text
\usepackage{xcolor}

% irgendwas mit figures
\usepackage{subcaption}

\begin{document}
\thispagestyle{fancy}

\section*{Aufgabe 1}
a) 

\newpage
\setlength{\headheight}{0cm}
\section*{Aufgabe 2}
a) Das Dipolmoment und seine Ableiutungen lauten:
\begin{align*}
	\vec p(t) 
	= \vec p_0 e^{-i\omega t}
	\qquad
	\dot{\vec p}(t) 
	= -i\omega \ \vec p_0 e^{-i\omega t}
	\qquad
	\ddot{\vec p}(t) 
	= -\omega^2 \ \vec p_0 e^{-i\omega t}
\end{align*}
Dann lässt sich für die Divergenz des Vektorpotentials zeigen:
\begin{align*}
	\diver \vec A(\vec r, t)
	&= \diver \frac{\mu_0}{4\pi r} \dot{\vec p} \left(t - \frac rc\right) \\
	% p einsetzen
	&= \diver \frac{-i \omega \mu_0}{4\pi r} \vec p_0 \exp \left( i\omega \frac rc - i\omega t \right) \\
	% divergenz nach vorne
	&= 
		\frac{-i \omega \mu_0}{4\pi}  \
		\diver \vec p_0 \frac{\exp\left(i\omega\frac rc - i\omega t\right)}{r} \\
	% produktregel
	&= 
		\frac{-i \omega \mu_0}{4\pi}  \
		\Bigg (
			\frac{\exp\left(i\omega\frac rc - i\omega t\right)}{r}  
			\underbrace{\diver \vec p_0}_{=0}
			+ \vec p_0 \cdot \vec \nabla \frac{\exp\left(i\omega\frac rc - i\omega t\right)}{r}
		\Bigg ) \\
	% 0 Term rausstriechen
	&= 
		\frac{-i \omega \mu_0}{4\pi}  \
		\Bigg (
			\vec p_0 \cdot \vec \nabla \frac{\exp\left(i\omega\frac rc - i\omega t\right)}{r}
		\Bigg ) \\
	% divergenz einsetze
	&= 
		\frac{-i \omega \mu_0}{4\pi}  \ \vec e_r \cdot \vec p_0 \cdot
		 \del_r \frac{\exp\left(i\omega\frac rc - i\omega t\right)}{r} \\
	% nach r ableiten
	&= 
		\frac{-i \omega \mu_0}{4\pi}  \ \vec e_r \cdot \vec p_0 \cdot
		 \frac{r 
		 	\frac{i\omega}{c} \exp\left(i\omega\frac rc - i\omega t\right)
		 	- \exp\left(i\omega\frac rc - i\omega t\right)}{r^2} \\
	% bruch trennen, r statt e_r
	&= 
		\frac{-i \omega \mu_0}{4\pi}  \ \vec r \cdot 
		\left(
		 \frac{\vec p_0 i\omega \exp\left(i\omega\frac rc - i\omega t\right)}{c r^2}
		 - \frac{\vec p_0 \exp\left(i\omega\frac rc - i\omega t\right)}{r^3}
		\right) \\
	% iomega reinziehen
	&= 
		- \frac{\mu_0}{4\pi}  \ \vec r \cdot 
		\Bigg(
		 \underbrace{\frac{-\omega^2 \vec p_0 \exp\left(i\omega\frac rc - i\omega t\right)}{c r^2}}
		 _{\frac{\ddot{\vec p}(t - r/c)}{cr^2}}
		 - \underbrace{\frac{i \omega \vec p_0 \exp\left(i\omega\frac rc - i\omega t\right)}{r^3}}
		 _{- \frac{\dot{\vec p}(t - r/c)}{r^3}}
		\Bigg) \\
	% p einsetzen
	&= 
		- \frac{\mu_0}{4\pi}  \ \vec r \cdot 
		\left(
			\frac{\ddot{\vec p} \left(t - \frac rc \right)}{cr^2} 
			+ \frac{\dot{\vec p}\left( t - \frac rc \right)}{r^3}
		\right) \\
	% epsz einsetzen
	&= 
		- \frac{\mu_0\epsz}{4\pi\epsz}  \ \vec r \cdot 
		\left(
			\frac{\ddot{\vec p} \left(t - \frac rc \right)}{cr^2} 
			+ \frac{\dot{\vec p}\left( t - \frac rc \right)}{r^3}
		\right) \\
	% nach c umstellen
	&= 
		- \frac{1}{c^2} \frac{1}{4\pi\epsz}  \ \vec r \cdot 
		\left(
			\frac{\ddot{\vec p} \left(t - \frac rc \right)}{cr^2} 
			+ \frac{\dot{\vec p}\left( t - \frac rc \right)}{r^3}
		\right) \\
	\overset{!}&{=} - \frac{1}{c^2} \dot \phi(\vec r, t) \\
	% nach dot phi uumstellen
	\Rightarrow \
	\dot \phi(\vec r, t)
	&= 
		\frac{1}{4\pi\epsz}  \ \vec r \cdot 
		\left(
			\frac{\ddot{\vec p} \left(t - \frac rc \right)}{cr^2} 
			+ \frac{\dot{\vec p}\left( t - \frac rc \right)}{r^3}
		\right) \\
	\Rightarrow \
	\phi(\vec r, t)
	&= 
		\frac{1}{4\pi\epsz}  \ \vec r \cdot 
		\left(
			\frac{\dot{\vec p} \left(t - \frac rc \right)}{cr^2} 
			+ \frac{\vec p\left( t - \frac rc \right)}{r^3}
		\right) + const.
\end{align*}

\end{document}
