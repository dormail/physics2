\documentclass[11pt a4paper]{article}
\usepackage[margin=2cm]{geometry}
\usepackage{amsmath, amssymb}
\usepackage{graphicx}
\usepackage{float}
\usepackage{aligned-overset}
\usepackage{subcaption}

% partielle ableitungen
\newcommand{\delr}{\partial_r}
\newcommand{\deltheta}{\partial_\theta}
\newcommand{\delphi}{\partial_\varphi}

% elektrische feldkonstante
\newcommand{\epsz}{\epsilon_0}
% 1 / 4pi eps
\newcommand{\kco}{\frac{1}{4\pi\epsilon_0}}

% rotation
\newcommand{\rot}{\text{rot}}

% fancy header
\usepackage{fancyhdr}
\fancyhf{}
% vspaces in den headern fuer Distanzen notwendig
% linke Seite: Namen der Abgabegruppe
\lhead{\textbf{Matthias Maile\\Roman Surma}\vspace{1.5cm}}
% rechte Seite: Modul, Gruppe, Semester
\rhead{\textbf{Physik II - Gruppe 2\\Sommersemester 2020}\vspace{1.5cm}}
% Center: nr. des blattes
\chead{\vspace{2.5cm}\huge{\textbf{18. Übungsblatt}}}
% benoetigt damit der eigentliche Text nicht in der Überschrift steckt
\setlength{\headheight}{4cm}

% zum zeichnen tikz
\usepackage{tikz}

% fuer fabigen text
\usepackage{xcolor}

\begin{document}
\thispagestyle{fancy}
\section*{Aufgabe 2}
Biot-Savar'sches Gesetz:
\[
	\mathbf B(\mathbf r) = \frac{\mu_0}{4\pi} I \int \frac
	{d\mathbf l \times (\mathbf r - \mathbf{r^\prime})}
	{\vert \mathbf r - \mathbf{r^\prime}\vert^3}
\]
Zur einfacheren Berechnung teilen wir den Draht in die zwei geraden 
Segmente (welche aufgrund der Symmetrie das Gleiche Feld am Punkt $P$ 
besitzen) und in das Halbkreissegment auf. \\
Zur Verinfachung wird das Koordinatensystem so gelegt, dass der Punkt 
$P$ im Urspung liegt. 
\newline
Berechnung des Halbkreissegements:
\begin{align*}
	\mathbf B_{HK}(\mathbf r) 
	&= \frac{\mu_0}{4\pi} I \int \frac
	{d\mathbf l \times (\mathbf r - \mathbf{r^\prime})}
	{\vert \mathbf r - \mathbf{r^\prime}\vert^3} \\
	% parametrisierung und zylinder koordianten
	&= \frac{\mu_0}{4\pi} I
	\int_0^\infty \int_{-\infty}^\infty \int_0^{2\pi}
	\frac
	{\vec e_\varphi \times (\mathbf r - \mathbf{r^\prime})}
	{\vert \mathbf r - \mathbf{r^\prime}\vert^3}
	\cdot r^\prime \cdot
	\delta(z^\prime) \cdot \delta(r^\prime - R) \cdot 
	\Theta(\pi - \varphi^\prime)
	\ d\varphi^\prime dz^\prime d\rho^\prime \\
	% r = ursprung
	&= \frac{\mu_0}{4\pi} I
	\int_0^{\pi}
	\frac
	{\vec e_\varphi \times \mathbf{r^\prime}}
	{\vert \mathbf{r^\prime}\vert^3} \cdot R
	\ d\varphi^\prime \\
	% kuerzen
	&= \frac{\mu_0}{4\pi} I
	\int_0^{\pi} \frac
	{\vec e_\varphi \times \vec e_\rho}
	{R^2}
	\cdot R
	\ d\varphi^\prime \\
	% kreuzprodukt ausrechnen
	&= \frac{\mu_0}{4\pi} I
	\int_0^{\pi} \frac
	{\vec e_z}
	{R}
	\ d\varphi^\prime \\
	% integrieren und kuerzen
	&= \frac{\mu_0 \ I}{4R} \vec e_z
\end{align*}
Die Berechnung des geraden Leiterelements erfolgt mit der Formel für das
Magnetfeld des endlichen Leiters, in Abhängigkeit von den  Winkeln an den 
Enden:
\begin{align*}
	B_{ger} (\mathbf r)
	&= \frac{\mu_0 \ I}{4 \pi R} 
	(\sin \theta_2 - \sin \theta_1) \\
	% winkel einsetzen
	&= \frac{\mu_0 \ I}{4 \pi R} 
	\left( \sin \frac\pi2 - \sin 0 \right) \\
	% ausrechnen
	&= \frac{\mu_0 \ I}{4\pi R}
\end{align*}
Das Magnetfeld folgt dann:
\[
	\mathbf B(\mathbf r) = \mathbf B_{HK}(\mathbf r) +
	2 \ \mathbf B_{ger} (\mathbf r)
	= \frac{\mu_0 \ I}{4R} \vec e_z + \frac{\mu_0 \ I}{2\pi R} \vec e_z
	= \frac{\pi + 2}{4\pi R} \mu_0 \ I \ \vec e_z
	\]

\newpage
\setlength{\headheight}{0cm}
\section*{Aufgabe 3}
Das magnetische Feld kann mit dem Biot-Savart Gesetz ermittelt werden:
\begin{align*}
	\mathbf B(\mathbf r) 
	&= \frac{\mu_0}{4\pi} \cdot I \cdot \int 
	\frac{d \mathbf{l} \times (\mathbf{r} - \mathbf{r^\prime})}
	{\vert \mathbf{r} - \mathbf{r^\prime} \vert^3} \\
	% parametrisierung des leiters
	&= \frac{\mu_0}{4\pi} \cdot I \cdot 
	\int_0^\infty \int_{-\infty}^\infty \int_0^{2\pi}
	\frac{\vec e_\varphi \times 
	(\mathbf{r} - \mathbf{r^\prime})}
	{\vert \mathbf{r} - \mathbf{r^\prime} \vert^3} \cdot
	\delta(R - \rho^\prime) \cdot \delta(z^\prime) \cdot \rho^\prime
	\ d\rho^\prime \ dz^\prime \ d\varphi^\prime \\
	% delta funktionen integrieren
	&= \frac{\mu_0}{4\pi} \cdot I \cdot 
	\int_0^{2\pi}
	\frac{ \vec e_\varphi \times 
	(z \ \vec e_z - r^\prime \vec e_\rho)}
	{\left( R^2 + z^2 \right)^{1.5} } \cdot
	R
	\ d\varphi^\prime \\
	% kreuzprodukte ermitteln
	&= \frac{\mu_0}{4\pi} \cdot I \cdot 
	\int_0^{2\pi}
	\frac{z \ \vec e_\rho + R \ \vec e_z}
	{\left( R^2 + z^2 \right)^{1.5} } \cdot
	R
	\ d\varphi^\prime
\end{align*}
Da sich der Magnetische Dipol auf der $z$-Achse befindet, verschwindet der
$\vec e_\rho$.
\begin{align*}
	\Rightarrow
	\mathbf B(\mathbf r) 
	&= \frac{\mu_0}{4\pi} \cdot I \cdot 
	\int_0^{2\pi}
	\frac{R^2 \ \vec e_z}
	{\left( R^2 + z^2 \right)^{1.5} }
	\ d\varphi^\prime \\
	% nach phi integrieren
	&= \frac{\mu_0 \ I}{2}
	\frac{R^2 \ \vec e_z}
	{\left( R^2 + z^2 \right)^{1.5} }
\end{align*}
Daraus folgt die Kraft auf ein magnetisches Dipol $m$:
\begin{align*}
	\mathbf F 
	&= m_z \ \partial_z B_z \ \vec e_z \\
	% einsetzen
	&= m_z \ 
	\frac{R^2 \ \mu_0 \ I}{2}
	\ \partial_z 
	\left( R^2 + z^2 \right)^{-1.5}
	\vec e_z \\
	% ableiten
	&= m_z \ 
	\frac{R^2 \ \mu_0 \ I}{2}
	\ (-1.5)
	\left( R^2 + z^2 \right)^{-2.5} \cdot 2z \
	\vec e_z \\
	% bisschen ordentlicher hinschreiben
	&= \frac{-1.5 z \cdot m_z \ R^2 \ \mu_0 \ I}
	{(R^2 + z^2)^{2.5}} \ \vec e_z
\end{align*}
Für kleine Auslenkungen, also $z \ll R$ vereinfacht sich die Kraft:
\begin{align*}
	z \ll R \Rightarrow R^2 + z^2 \approx R^2 \Rightarrow
	\mathbf F
	&= \frac{-1.5 z \cdot m_z \ R^2 \ \mu_0 \ I}
	{R^5} \ \vec e_z \\
	% kuerzen
	&= \frac{-1.5 z \cdot m_z \ \mu_0 \ I}
	{R^3} \ \vec e_z
\end{align*}
Schreibt man diesen Ausdruck als Differentialgleichung erhalten wir den
harmonischen Oszillator:
\begin{align*}
	F_z = M\ddot z 
	&= \frac{-1.5 z \cdot m_z \ \mu_0 \ I}
	{R^3} \Rightarrow
	\ddot z + \underbrace{\frac{1.5\cdot m_z \ \mu_0 \ I}{M \cdot R^3}}
	_{\omega^2} \ z = 0
\end{align*}
Mit der uns schon bekannten Lösung:
\[
	z(t) = z(0) \cos(\omega t) + \dot z(0) \sin(\omega t)
\]

\newpage
\section*{Aufgabe 4}
\[
	\vec B(\vec r) = B \vec e_z, \ B = const.
\]
a) gesucht sei ein $\vec A$, sodass
\[ \rot \vec A = \vec B \]

Mit dem Ansatz 
\[
	\vec A = M \cdot \vec r, \quad
	\vec r = (x,y,z)^T, \
	M \in \mathbb{R}^{3\times3}
\]

sind die Einträge von $M$ die partiellen Ableitungen von $\vec A$:
\[ M_{ij} = \partial_j A_i \]

Aus der Vorraussetzung $\rot \vec A =\vec B$ folgen die Bedinungen für 
$M$:
\begin{align*}
	\rot \vec A = \begin{pmatrix}
		M_{3,2} - M_{2,3} \\
		M_{1,3} - M_{3,1} \\
		M_{2,1} - M_{1,2} 
	\end{pmatrix}
	\overset{!}{=}
	\begin{pmatrix} 0 \\ 0 \\ B \end{pmatrix}
	\ \Rightarrow \
	M_{3,2} = M_{2,3}, \
	M_{1,3} = M_{3,1}, \
	M_{2,1} = M_{1,2} +B
\end{align*}

Daraus folgt dann die Matrix $M$ (mit den Freiheitsgraden $\alpha$, ...):
\[
	M = \begin{pmatrix}
		\alpha 		&\beta 		&\gamma \\
		\beta + B 	&\delta 	&\epsilon \\
		\gamma 		&\epsilon 	&\zeta
	\end{pmatrix}
\]

Das einfachste Vektorfeld für $\vec A$ wäre das wo alle Freiheitsgrade auf 
0 gesetzt werden:
\[
	\vec A_0 = M_0 \cdot \vec r = 
	M = \begin{pmatrix}
		0 &0 &0 \\ \beta &0 &0 \\ 0 &0 &0
	\end{pmatrix}
	\cdot \begin{pmatrix}
		x \\ y \\ z
	\end{pmatrix}
	= 
	\begin{pmatrix}
		0 \\ Bx \\0
	\end{pmatrix}
\]


b) Wir wählen zwei Lösungen $\vec A_1$ und $\vec A_2$:
\[
	\vec A_1 = M_1 \cdot \vec r = \begin{pmatrix}
		1 &0 &0 \\
		\beta &0 &0 \\
		0 &0 &0 \\
	\end{pmatrix}
	\cdot \vec r
	= \begin{pmatrix}
		x \\Bx \\0
	\end{pmatrix}
\]




\end{document}
